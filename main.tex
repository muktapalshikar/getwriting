\documentclass{getwriting}
\title{\lstinline{getwriting}: A Minimal~\LaTeX~Template For Journal Articles}
\author[1,2]{Notice the ORCID links, PhD \orcidlink{xxxx-xxxx-xxxx-xxxx} \thanks{firstauthor@famoustown.edu}}
\author[3]{Emails are in the footnote \orcidlink{xxxx-xxxx-xxxx-xxxx} \thanks{secondauthor@historytown.edu}}
\author[1,3]{Usually only this person's email needs to be shown, PhD \orcidlink{xxxx-xxxx-xxxx-xxxx} \thanks{thirdauthor@lessfamoustown.edu}}
%usually only the corresponding authors' emails are on the manuscript, but you can add all if you want
\affil[1]{\footnotesize Department of Science, University of Famoustown}
\affil[2]{\footnotesize Department of Mathematics, University of Lessfamoustown}
\affil[3]{\footnotesize Department of History, University of Historytown}
\begin{document}
\maketitle
\begin{abstract}
\noindent This is an example of the document that can be generated with the \lstinline{getwriting} class. I usually use this template or minor variations thereof to write manuscripts.\\
\noindent This is a tidy, minimal, readable, \textit{good enough} format that has everything you need and nothing you don't. I offer a quick start to writing articles by including commonly used packages and a template that matches the general format of most journal articles. I have kept available options to the bare minimum. If you really want to tweak something, edit the cls file. If you need to reformat your document for journals later, it's easy to copy and paste from this template as there are no custom sections.\\
\noindent \textbf{Don't waste time formatting your document, just start writing.}
\\\\
\textbf{Keywords:} Article templates, Overleaf, minimal, bioR$\chi$v, medR$\chi$v, format-free submissions,~\LaTeX

\end{abstract}
\section{Introduction}
\textbf{How to use this template: }
\begin{outline}[enumerate]
\1 \textbf{If you got this template from the Overleaf Gallery:}
Easy, just edit as necessary. Keep this tex file around for reference. Use a copy of this file to get started writing, and fill in all the sections. \textbf{I used the XeLaTex compiler}. 
\1 \textbf{If you didn't get this template from the Overleaf Gallery:}
\2 If you're a relative newcomer to \LaTeX and/or use Overleaf:
Import all files in this repository into Overleaf (yes, Overleaf, since it has the best package manager and managing~\LaTeX~packages on a personal installation is a nightmare and \textit{\textbf{you should be writing}}. Edit a copy of this tex file as described above.
\2 \textbf{If you use a local Tex installation:}
You can clone the GitHub repository from \url{https://github.com/mgp13/getwriting} and use as needed. Alternatively, you can fork the GitHub repository as a template.
\end{outline}
\par
This document uses the beautiful and free Atkinson Hyperlegible font, which is designed for low-vision readers. You should consider doing the same. The TrueType font files are included with this template and you can download the font from the Braille Institute here: \url{https://brailleinstitute.org/freefont}.
\par
Previous work has shown something, but not this, and I have cited a paper~\cite{scbonita}. 
\par
Look, this is a new paragraph. \textbf{And here is some bold text.} I can even \textit{italicize} text.
\par
I can refer to subsequent sections and elements like so: \hyperref[results]{\autoref{results}} and modify the display text like so: \hyperref[fig:figure1]{Look at figure 1!}.
\par
\subsection{Disclaimers}
This is a numbered subsection, by the way.
\par
\textbf{This template is provided as-is} and I take no responsibility for any errors/disasters. I'm not a~\LaTeX~expert even though I can typeset~\LaTeX~and I can even cite papers~\cite{scbonita} and cross-reference equations~\hyperref[eq:refname]{(\autoref{eq:refname})}. Play with the .cls file, look it up on StackOverflow and/or the Overleaf documentation, do whatever you like.
\section{Results}\label{results}
\subsection{Citations and cross-references}
Here are some magnificent results. Previous work has shown something, but not this, and I have cited a paper~\cite{scbonita}. 
\par
Here's how to cross-reference a table:~\hyperref[tab:table1]{\autoref{tab:table1}}
\par
Here is how I would typeset the name of a software package: \lstinline{software}
\subsection{All About Figures}
Here is even more data. I can reference a figure to support my point \hyperref[fig:figure1]{\autoref{fig:figure1}}.
You can find this figure in \hyperref[figures]{\autoref{figures}}. You can even refer to a supplementary figure: \hyperref[fig:supplementaryfigure1]{\autoref{fig:supplementaryfigure1}}. Note the difference in the figure title format. This is set by a renewcommand line in the \hyperref[suppinfo]{Supplementary Information section}.
\par
Speaking of figures, here are the rules for this minimal template. Most of these rules apply to tables as well apart from the ones that don't.
\begin{outline}
    \1 Figures must all be in a section at the end. Don't bother trying to have the figures near where you reference them. This is a futile, time- consuming exercise \& \textbf{you should be writing, not formatting}. Notice how I formatted the ampersand (\&).
    \1 Each figure must start on a new page. This is the only clean way.
\end{outline}

\section{Materials and Methods}
\section{How to Format}
\subsection{All About Text Formatting}
You should almost never highlight text, but here's how you can: \hl{this is a highlighted phrase}.\\
Consider \textit{italics}, \textbf{bold} or \emph{emphasis} instead.\\
Write inline math like this ($\sqrt{-1}$), Greek letters like this ($\alpha$), inline math and Greek together like this ($\alpha + \beta = \gamma$), and basic chemical formulae or any text requiring super/sub-scripts in math mode like this $O_2$, $O^2$, $O_{1,2,3}$. 
\subsection{All About Equations}
I can refer to equation eq:refname like so: this is some text about \hyperref[eq:refname]{\autoref{eq:refname}} and I can also modify display text in the hyperlink like this \hyperref[eq:refname]{(\autoref{eq:refname} This is a string)}
\begin{equation}
	\cos^3 \theta =\frac{1}{4}\cos\theta+\frac{3}{4}\cos 3\theta
	\label{eq:refname}
\end{equation}
\section{All About Lists}
I can list things like this:
\begin{enumerate}
	\item First item in a list
	\begin{enumerate}
	    \item This is a subitem.
	\end{enumerate}
	\item Second item in a list
\end{enumerate}
Or even like this:
\begin{itemize}
	\item{First item in a list}
	\item{Second item in a list}
\end{itemize}
\textit{\textbf{But I really prefer \textbf{multi-level lists} like this:}}
I have pre-set the \lstinline{enumerate} package options to show Arabic numerals for level 1, lower-case alphabet for level 2, and lower-case Roman numerals for level 3. This is a sensible default and you're unlikely to gain much from changing them.
\begin{outline}[enumerate]
	\1 First item in an outline
	\2 Second item in an outline
	\3 Third item in an outline
	\1 Fourth item in an outline
	\2 Fifth item in an outline
\end{outline}
\section{Discussion}
All rules are meant to be broken (for good reasons). Use this template as just that, a \textit{template}. 
\par
\begin{center}
    \Large{\textit{\textbf{Happy writing!}}}
\end{center}
\newpage
\section{Figures}\label{figures}
\begin{figure}[H]
\label{fig:figure1}
\includegraphics[width=5.25in]{figures/testPicture.jpg}
\caption{\textbf{Here is the title of my caption} Here is text describing each panel in this figure. This is a photograph by Bob Brewer and incidentally this is how you typeset a URL: \url{https://unsplash.com/photos/sFsumfD7Pbs}}
\end{figure}
\newpage
\section{Tables}
Rules for tables:
\begin{outline}
\1 Each table goes on a new page. All tables should be in this section.
\1 I only use the longtables and booktabs packages, regardless of the length of the tables. These produce beautiful and readable tables. It's easy to generate longtable code from a CSV with Pandas: \url{https://pandas.pydata.org/docs/reference/api/pandas.DataFrame.to_latex.html}
\1 The makecell command allows multi-lined cells.
\end{outline}
\par
An example table~\hyperref[tab:table1]{(\autoref{tab:table1})} will start on the next page.
\newpage
\begin{longtable}{lll}
\caption{Here's a table made with \lstinline{longtable} and \lstinline{booktabs}. This is the loooooong caption.}
\label{tab:table1}\\
\toprule
\makecell{\makecell{Column 1}} & \makecell{\makecell{Column 2}} & \makecell{\makecell{Column 3}} \\ \\
\midrule
\endfirsthead
\caption[]{This is the SHORT caption that is shown when a table continues to the next page} \\
\toprule
\makecell{\makecell{Column 1}} & \makecell{\makecell{Column 2}} & \makecell{\makecell{Column 3}} \\ \\
\midrule
\endhead
\midrule
\multicolumn{3}{r}{{Continued on next page}} \\
\midrule
\endfoot
\bottomrule
\endlastfoot
          \makecell{Row 1} &  \makecell{something} &  \makecell{anything}\\
          \makecell{Row 2} &   \makecell{something} &  \makecell{anything}\\
            \makecell{Row 3} &   \makecell{something} &  \makecell{anything}\\
                      \makecell{Row 1} &  \makecell{something} &  \makecell{anything}\\
          \makecell{Row 2} &   \makecell{something} &  \makecell{anything}\\
            \makecell{Row 3} &   \makecell{something} &  \makecell{anything}\\
                      \makecell{Row 1} &  \makecell{something} &  \makecell{anything}\\
          \makecell{Row 2} &   \makecell{something} &  \makecell{anything}\\
            \makecell{Row 3} &   \makecell{something} &  \makecell{anything}\\
                      \makecell{Row 1} &  \makecell{something} &  \makecell{anything}\\
          \makecell{Row 2} &   \makecell{something} &  \makecell{anything}\\
            \makecell{Row 3} &   \makecell{something} &  \makecell{anything}\\
                      \makecell{Row 1} &  \makecell{something} &  \makecell{anything}\\
          \makecell{Row 2} &   \makecell{something} &  \makecell{anything}\\
            \makecell{Row 3} &   \makecell{something} &  \makecell{anything}\\
                      \makecell{Row 1} &  \makecell{something} &  \makecell{anything}\\
          \makecell{Row 2} &   \makecell{something} &  \makecell{anything}\\
            \makecell{Row 3} &   \makecell{something} &  \makecell{anything}\\
                      \makecell{Row 1} &  \makecell{something} &  \makecell{anything}\\
          \makecell{Row 2} &   \makecell{something} &  \makecell{anything}\\
            \makecell{Row 3} &   \makecell{something} &  \makecell{anything}\\
                      \makecell{Row 1} &  \makecell{something} &  \makecell{anything}\\
          \makecell{Row 2} &   \makecell{something} &  \makecell{anything}\\
            \makecell{Row 3} &   \makecell{something} &  \makecell{anything}\\
                      \makecell{Row 1} &  \makecell{something} &  \makecell{anything}\\
          \makecell{Row 2} &   \makecell{something} &  \makecell{anything}\\
            \makecell{Row 3} &   \makecell{something} &  \makecell{anything}\\
\end{longtable}
\newpage
\section*{Funding}
Thanks for the money.
\section*{Acknowledgements}
Shoutout to all my friends. 
\section*{Abbreviations}
\textbf{HIV}: human immunodeficiency virus; \textbf{DNA}: deoxyribonucleic acid
\newpage
\section*{Supplementary Information}
\label{suppinfo}
Supplementary Information should start on a new page. \\
The redefinitions below will make sure that your supplementary figures and tables are referred to as 'Supplementary Figure X' and 'Supplementary Table Y' respectively.
\setcounter{figure}{0}    
\renewcommand{\figurename}{Supplementary Figure }
\makeatletter
\def\fnum@figure{\figurename\thefigure}
\makeatother
\renewcommand{\tablename}{Supplementary Table }
\newpage
\subsection*{Supplementary Figures}
\begin{figure}[H]
\label{fig:supplementaryfigure1}
\begin{center}
\includegraphics[width=3.25in]{figures/supplementaryTest.jpg}
\end{center}
\caption{\textbf{Here is a VERTICAL figure that I have CENTERED.} Here is text describing each panel in this figure. This is a photograph by Andreas Dress and you can find the original here: \url{https://unsplash.com/photos/NNe6epzHGm8}}
\end{figure}
\newpage
\subsection*{Supplementary Tables}
Add some tables here.
\newpage
\bibliography{getwriting}
\end{document}
